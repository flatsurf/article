\documentclass{article}

\usepackage{amsmath,amssymb}
\usepackage{graphicx,subcaption}
\usepackage{url}
\usepackage[margin=2cm]{geometry}
\usepackage[abbreviations]{foreign}
\usepackage{xcolor}

\def\RR{\mathbb{R}}
\def\CC{\mathbb{C}}

\def\cA{\mathcal{A}}
\def\cE{\mathcal{E}}

\def\Re{\operatorname{Re}}
\def\Im{\operatorname{Im}}
\def\Sym{\operatorname{Sym}}

\newtheorem{definition}{Definition}
\newtheorem{remark}[definition]{Remark}
\newtheorem{lemma}[definition]{Lemma}
\newtheorem{theorem}[definition]{Theorem}

\title{Algorithms for translation surfaces}
\author{Vincent Delecroix for Julian R\"uth}

\newcommand{\commv}[1]{{\color{red!50!gray}[#1]}}	

\begin{document}
\maketitle

\section{Translation structures and measured foliations}

\subsection{Whitehead moves}

%%%%%%%%%%%%%%%%%%%%%%%%%%%%%%%%%%%%%%%%%%%%%%%%%%%%%%%%%%%%%%%%%%%%%%%%%%%%%%%%%
\section{Topology and combinatorial maps}
\label{sec:TriangulationsAndTranslationStructures}

A cell decomposition of a surface can be encoded by a triple of
permutations.
\begin{definition}
Let $\cE$ be a finite set.
A \emph{combinatorial map} or a \emph{fat graph} on $\cE$ is a triple
of permutations $(\sigma_v, \sigma_e, \sigma_f) \in \Sym(\cE)^3$ on
the set $\cE$ such that
\begin{enumerate}
\item $\sigma_f \circ \sigma_e \circ \sigma_v = id_\cE$,
\item $\sigma_e$ is an involution without fixed points,
\item the group generated by $\sigma_v, \sigma_e, \sigma_f$ acts transitively on $\cE$.
\end{enumerate}
\end{definition}
In the definition the letters  $v$, $e$ and $f$ respectively stand for vertices, edges
and faces.  Each cycle of the cycle decomposition of $\sigma_v$, $\sigma_e$
and $\sigma_e$ represents respectively a vertex, an edge
and a face of the cell decomposition. In particular, the cardinality of $\cE$
is even. The transitivity of the group ensures the connectivity of the surface.
The set $\cE$ can be identified with half edges or oriented edges.

We will always denote $n_v$, $n_e$ and $n_f$ respectively the number of vertice,
edges and faces of the combinatorial map $(\sigma_v, \sigma_e, \sigma_f)$. In
combinatorial terms it corresponds to the number of cycles in the cycle decomposition
of $\sigma_v$, $\sigma_e$ and $\sigma_f$.

A particular case of combinatorial maps are \emph{triangulations}
when the permutation $\sigma_f$ is a product of disjoint 3-cycles.
In these situation $n$ is a multiple of $6$.

\begin{figure}[!ht]
\begin{minipage}{0.4\textwidth}
\begin{center}
\includegraphics{pictures/Torus.pdf}
\subcaption{A square torus $\sigma_v = (0,1,2,3)$, $\sigma_e=(0,2)(1,3)$, $\sigma_f=(0,1,2,3)$.}
\end{center}
\end{minipage}
\hspace{0.1\textwidth}
\begin{minipage}{0.4\textwidth}
\begin{center}
\includegraphics{pictures/TriangulatedTorus.pdf}
\subcaption{A triangulated torus $\sigma_v = (0,4,1,2,5,3)$, $\sigma_e=(0,2)(1,3)(4,5)$, $\sigma_f=(0,1,5)(2,3,4)$.}
\end{center}
\end{minipage}
\caption{Examples of combinatorial maps on a torus.}
\label{fig:CombinatorialMapTori}
\end{figure}


The \emph{dual} of the combinatorial map $(\sigma_v, \sigma_e, \sigma_f)$ is the combinatorial
map $(\sigma_f^{-1}, \sigma_e, \sigma_v^{-1})$.
\begin{figure}[!ht]
\begin{center}
\includegraphics{pictures/DualCombinatorialMaps.pdf}
\end{center}
\caption{Dual combinatorial maps}
\label{fig:DualCombinatorialMaps}
\end{figure}

Figure~\ref{fig:DualCombinatorialMaps} provides a local picture of two dual
combinatorial maps.

%%%%%%%%%%%%%%%%%%%%%%%%%%%%%%%%%%%%%%%%%%%%%%%%%%%%%%%%%%%%%%%%%%%%%%%%%%%%%%%%%%%%%%%%%%%%%%%
\section{Elementary moves: removing, adding, flipping edges}
\label{sec:ElementaryMoves}
Let $(\sigma_v, \sigma_e, \sigma_f)$ be a combinatorial map on $\cE$. We define three operations
(and their inverses) that allow to move through the space of all combinatorial maps of given genus.

\subsection{Removing edges / splitting faces}
\textbf{removing an edge}:
Let $(x,y)$ be a transposition of $\sigma_e$ such that the two faces adjacent to this
edge are different then we can remove this edge and obtain a new combinatorial map.
\begin{center}\includegraphics{pictures/RemoveEdge.pdf}\end{center}
The operation on the permutations is given as follows
\begin{align*}
\sigma_v = (j,x,A) (i,y,B) \ldots & \mapsto (j,A) (i,B) \ldots \\
\sigma_e = (j, \sigma_e(j)) (i, \sigma_e(i)) (x,y) \ldots & \mapsto (j, \sigma_e(j)) (i, \sigma_e(i)) \ldots \\
\sigma_f = (x,i,C) (y,j,D) \ldots & \mapsto (i,C,j,D) \ldots 
\end{align*} 
This operation merges the two faces adjacent to the edge $(x,y)$ into a single bigger face.
The new combinatorial map is on the set $\cE \setminus \{x,y\}$.

\textbf{splitting a face}:
This is simply the reverse procedure of removing an edge.

\subsection{Contracting edges / splitting vertices}
\textbf{contracting an edge}:
dual to "removing an edge"

\textbf{splitting a vertex}:
reverse to the above.

\subsection{Forward flip / backward flip}
The operation considered above modifies the number of edges of the combinatorial map.

\textbf{forward edge flip in a face}
\begin{center}\includegraphics{pictures/FlipEdge.pdf}\end{center}
Given an edge $(x,y)$ that is adjacent to distinct faces, we can perform an
"edge flip" as in the above figure.
The flip operation does not change the underlying set $\cE$.

\textbf{forward edge flip at a vertex}

\bigskip

\section{Translation structures}
\begin{definition}
\label{def:FlatTriangulation}
Let $T = (\sigma_v, \sigma_e, \sigma_f) \in \Sym(\cE)^3$ be a 
triangulation, a \emph{translation structure} on $T$
is a family of non-zero complex numbers $\{\zeta_i\}_{i\in\cE}$
such that
\begin{enumerate}
\item for each cycle $(i,j)$ of $\sigma_e$ we have $\zeta_i + \zeta_j = 0$
\item for each cycle $(i_1,i_2,i_3)$ of $\sigma_f$ 
the vectors $\zeta_{i_1}$, $\zeta_{i_2}$, $\zeta_{i_3}$ form the
sides of a triangle in $\CC$ where the sides appear in counter-clockwise order
(in particular $\zeta_{i_1} + \zeta_{i_2} + \zeta_{i_3} = 0$).
\end{enumerate}
\end{definition}
A translation surface is a geometrical notion and using triangulations
and the above definition is very often not enough.
For example in~\cite{veerer}, we need a more restrictive version
(related to the slopes of the vectors $\zeta_i$).
On the other hand, one might want to generalize translation structures
to cell decompositions that are not triangulations.
For now, let us consider a simple definition that is not the most
general possible but which is simple to state.
\begin{definition}
\label{def:EmbeddedTranslationStructure}
Let $M = (\sigma_v, \sigma_e, \sigma_f) \in \Sym(\cE)$ be a combinatorial
map. An \emph{embedded translation structure} for $M$ is a zet of non-zero
complex numbers $\{\zeta_i\}_{i=1,\ldots,2n}$ such that
\begin{enumerate}
\item for each cycle $(i,j)$ of $\sigma_e$ we have $\zeta_i + \zeta_j = 0$
\item for each cycle $(i_1,i_2,\ldots,i_k)$ of $\sigma_f$ the vectors
$\zeta_{i_1}$, \ldots, $\zeta_{i_k}$ form the sides of a $k$-gon in
$\CC$ where the vectors appear in counter-clockwise order.
\end{enumerate}
\end{definition}

A \emph{sub-translation structure} is one that can be obtained by forgetting
some edges (see also Section~\ref{sec:ElementaryMoves}. Because a polygon can
always be triangulated we have the following.
\begin{lemma}
Given a combinatorial map endowed with translation structure $(M,\zeta)$, there
exists a triangulation with a translation structure $(T,\zeta')$ so that $M$ is
a sub-translation structure of $(T,\zeta')$.
\end{lemma}


\subsection{Elementary moves on translation structures}
Given a combinatorial map endowed with a translation structure $(M, \zeta)$
we want to extend the elementary moves of Section~\ref{sec:ElementaryMoves}
that were operating on $M$ to the pair $(M, \zeta)$. The geometry of $\zeta$
forbids certain moves to happend.

\textbf{removing an edge / splitting a face}
Removing an edge might result in a non-embedded translation structure.
Splitting a face is always possible.

\textbf{contracting an edge / splitting a vertex}
Operations not allowed.

\textbf{forward flip / backward flip}
To perform a flip, the new edge to be added must be contained in the union of
the two faces (or in other words a saddle connection on the translation surface).
In the special cases the two adjacent faces are triangles, the condition
for flipping the edge (which is one diagonal of the quadrilateral) is that the
quadrilateral is convex.
\begin{figure}[!ht]
\begin{center}\includegraphics{pictures/QuadrilateralFlip.pdf}\end{center}
\caption{A flip in a quadrilateral.}
\end{figure}

%%%%%%%%%%%%%%%%%%%%%%%%%%%%%%%%%%%%%%%%%%%%%%%%%%%%%%%%%%%%%%%%%%%%%%%%%%%%%%%%
\section{From translation structures to interval exchange transformation}
Given a translation surface we want to study the behavior of its vertical foliations
(the foliation in any direction is relevant, but since the surface can be rotated,
we assume that we deal with the vertical). The information about the vertical
foliation can be encoded in a one-dimensional object: the 
(multi-)\emph{interval exchange transformations} that are intimately related
to \emph{train tracks}.

\subsection{Multi-interval exchange transformations}
As with translation structures, interval exchange transformation are described as
a pair $(\pi, \lambda)$ where $\pi$ is a combinatorial data and $\lambda$ is
a continuous data.

An interval exchange is easily describe in words. Consider the interval
$I = [0,1]$. Cut it into finitely many open sub-intervals
$I_1^{top} = (0,x_1^{top})$, $I_2^{top} = (x_1^{top}, x_2^{top})$, \ldots,
$I_d^{top} = (x_{d-1}^{top}, 1)$. Rearrange the $d$ interval using a fixed
permutation $\pi$ provide a map on $[0,1]$ which is a piecewise translation.
This map is an interval exchange transformation.

\begin{figure}[!ht]
\begin{minipage}{0.5\textwidth}
\begin{center}\includegraphics{pictures/MultiIET.pdf}\end{center}
\subcaption{A multi-interval exchange transformation with $\cE^{top} = \{a,b,c,d\}$,
$\cE^{bot} = \{a',b',c',d'\}$ and $\sigma_e = (a,a')(b,b')(c,c')(d,d')$.
The multi-interval exchange maps the sub-interval $I_a$ to $I_{a'}$ by
translation and similarly for the other sub-intervals.}
\label{fig:MultiIET}
\end{minipage}
\end{figure}

Let us turn to the formal definition of multi-interval exchange transformations
where several intervals come into play. We first describe the combinatorial part
that we call multi-permutation.
\begin{definition}
Let $\cE$ be a finite set and $\sigma_e: \cE \to \cE$ be an involution without fixed point.
A \emph{multi-permutation} on $(\cE, \sigma_e)$ is the data of
\begin{itemize}
\item A partition $\cE = \cE^{top} \sqcup \cE^{bot}$ with $\sigma_e(\cE^{top}) = \cE^{bot}$
\item Two sub-partitions $\cE^{top} := \cE^{top}_1 \sqcup \cE^{top}_2 \sqcup \ldots \sqcup \cE^{top}_m$
and $\cE^{bot} := \cE^{bot}_1 \sqcup \cE^{bot}_2 \sqcup \cdots \sqcup \cE^{bot}_m$ into
$m$ non-empty sets
\item For each $j \in \{1, \ldots, m\}$, bijections $\pi^{top}_j: \cE^{top}_j \to \{1, \ldots, d_j^{top}\}$
and $\pi^{bot}_j: \cE^{bot}_j \to \{1, \ldots, d_j^{bot}\}$ where
$d_j^{top} := \# \cE^{top}_j$ and $d_j^{bot} := \# \cE^{bot}_j$.
\end{itemize}
\end{definition}

To represent a multi-permutation, we will use the notation
\[
\left(\begin{array}{c|c|c}
(\pi_1^{top})^{-1}(1)\ (\pi_1^{top})^{-1}(2)\ \ldots (\pi_1^{top})^{-1}(d_1^{top}) &
\ldots &
(\pi_m^{top})^{-1}(1)\ (\pi_m^{top})^{-1}(2)\ \ldots (\pi_m^{top})^{-1}(d_m^{top})
\\
(\pi_1^{bot})^{-1}(1)\ (\pi_1^{bot})^{-1}(2)\ \ldots (\pi_1^{bot})^{-1}(d_1^{bot}) &
\ldots &
(\pi_m^{bot})^{-1}(1)\ (\pi_m^{bot})^{-1}(2)\ \ldots (\pi_m^{bot})^{-1}(d_m^{bot})
\end{array}\right)
\]
On the example of Figure~\ref{fig:MultiIET} the multi-permutation is
\[
\left(\begin{array}{c|c}
a\ b & c\ d \\
c\ b\ d & a \\
\end{array} \right)
\]

\begin{definition}
Let $\pi = \{(\cE^{top}_j, \pi^{top}_j, \cE^{bot}_j, \pi^{bot}_j\}_{j=1,\ldots,m}\}$ be
a multi-permutation on $(\cE, \sigma_e)$. A \emph{length data} for $\pi$ is
an element $\lambda \in \RR_{> 0}^{\cE}$ such that
\begin{itemize}
\item for each $i \in \cE$, $\lambda_i = \lambda_{\sigma_e(i)}$
\item for each $j \in \{1,\ldots,m\}$, $\displaystyle \sum_{i = 1}^{d_j^{top}} \lambda_{(\pi^{top})^{-1}(i)} = \sum_{i = 1}^{d_j^{bot}} \lambda_{(\pi^{bot})^{-1}(i)}$.
\end{itemize}
\end{definition}


\begin{definition}
Let $\pi = \{(\cE^{top}_j, \pi^{top}_j, \cE^{bot}_j, \pi^{bot}_j\}_{j=1,\ldots,m}\}$
be a multi-permutation on $(\cE, \sigma_e)$ and $\lambda \in \RR^\cE$
a length data for $\pi$.

The \emph{multi-interval exchange transformation} associated to the pair $(\pi,\lambda)$
is the partial transformation on $I = I_1 \sqcup I_2 \sqcup \ldots \sqcup I_m$ where
\[
I_j := \left(0, \sum_{i=1}^{d_j^{top}} \lambda_{(\pi^{top}_j)^{-1}(i)} \right)
\]
defined as in Figure~\ref{fig:MultiIET}.
\end{definition}

\begin{remark}
An alternative definition of multi-permutation is to consider an alphabet $\cA$
with $\# \cA = \frac{1}{2} \# \cE$ that is in bijection with the orbit of
$\sigma_e$. That is we have bijections $p^{top}: \cA \to \cE^{top}$ and
$p^{bot}: \cA \to \cE^{bot}$ so that $\sigma_e \circ p^{top} = p^{bot}$.
But using the definition with $(\cE, \sigma_e)$ is more reminiscent of
the combinatorial map that we previously introduced.

This would match Yoccoz notations~\cite{Yoccoz}.
\end{remark}

\subsection{From triangulations with translation structures to multi-interval exchange transformations}
Let $M = (\sigma_v, \sigma_e, \sigma_f)$ be a triangulation on
the finite set $\cE$ and $\zeta \in \CC^\cE$ a translation structure for $M$.
We associate to this pair a multi-interval exchange on
$n_f$-intervals where $n_f$ is the number of triangles in $M$.

Each triangle $(i,j,k)$ in $M$ can be drawn in the plane and has a left-most and
a right-most vertex. This decomposes its sides into a top and a bot partition.
See Figure~\ref{fig:FromTrianglesToIntervals}
\begin{figure}[!ht]
\begin{minipage}{0.4\textwidth}
\begin{center}
\includegraphics{pictures/TriangleToInterval1.pdf}
\end{center}
\end{minipage}
\hspace{.1\textwidth}
\begin{minipage}{0.4\textwidth}
\begin{center}
\includegraphics{pictures/TriangleToInterval2.pdf}
\end{center}
\end{minipage}
\caption{From triangles to intervals.}
\label{fig:FromTrianglesToIntervals}
\end{figure}

Note that the length data of the multi-interval exchange transformation is
simply the vector of absolute values of real parts of the vectors $\zeta_i$.

\subsection{From multi-iet to iet}
We now describe an algorithm that given a multi-interval exchange transformation
produces an interval exchange transformation (that is a multi-interval exchange
transformation on a single interval).

The only tool we use is to perform induction similar to Rauzy induction.
\begin{definition}
Let
$\pi = \{(\cE^{top}_j, \pi^{top}_j, \cE^{bot}_j, \pi^{bot}_j\}_{j=1,\ldots,m}\}$
be a multi-permutation.
If for $j$ we have $d_j^{top} = 1$, that is $\cE^{top}_j = {i}$ consists of a single
element. Then we perform an \emph{aggregation along $i$} by removing the
$j$-th interval of the multi-permutation and replacing
$i' = \sigma(i)$ by $\cE^{bot}_j$.

Similarly, if $d_j^{bot} = 1$.
\end{definition}
At the level of multi-interval exchange transformations, the aggregation
consists in taking the first return map of the multi-interval
exchange transformation after removing the $j$-th interval. Note that this
operation is purely combinatorial.

\begin{definition}
Let
$\pi = \{(\cE^{top}_j, \pi^{top}_j, \cE^{bot}_j, \pi^{bot}_j\}_{j=1,\ldots,m}\}$
be a multi-permutation and $\lambda \in \RR_{>0}^{\cE}$ an associated
length data.

Let $j \in \{1,\ldots,m\}$ and consider the right-most subintervals of
the $j$-th interval, that is $i^{top} := (\pi^{top}_j)^{-1}(d_j^{top})$
and $i^{bot} := (\pi^{bot}_j)^{-1}(d_j^{bot})$.

The \emph{right Rauzy induction on the $j$-th component} consists in inducing
the multi-interval exchange transformation after removing the smallest of
the two sub-intervals corresponding to $i^{top}$ or $i^{bot}$.
\end{definition}


The algorithm is obtained by repeating
\begin{enumerate}
\item if there is any aggregation, do it
\item if there is no, perform Rauzy inductions until there is one
\end{enumerate}

It might happen that Performing Rauzy induction we detect some saddle connection,
that is we might have $\lambda_{i^{bot}} = \lambda_{i^{top}}$. In that case, the
saddle connection should be stored for future use and the two sub-intervals must be
fusionned.

%\subsection{What is a zippered rectangle?}
%\begin{definition}
%\label{def:WeakZipperedRectangle}
%Let $\cE$ be a finite set of cardinality $2d$.
%A \emph{topological zippered rectangle} is a combinatorial map $(\sigma_v, \sigma_e, \sigma_f) \in \Sym(\cE)^3$ such that
%\begin{enumerate}
%\item $\sigma_f$ is a $2d$-cycle (that is the cell decomposition consists of a single cell),
%\item there is a partition of $\cE^{top} \cup \cE^{bot} = \cE$ such that the 
%involutions $(\sigma_f)^d$ and $\sigma_e$ exchanges $\cE^{top}$ and $\cE^{bot}$.
%\end{enumerate}
%A \emph{weak zippered rectangle translation structure} for $(\sigma_v, \sigma_e, \sigma_f)$ is
%a set of complex non-zero vectors $\zeta_i$ such that
%\[
%\forall i \in \cE^{bot}, \Re(\zeta_i) \geq 0
%\qquad
%\forall i \in \cE^{top}, \Re(\zeta_i) \leq 0.
%\]
%\end{definition}
%Note that the above definition only involves the real parts of the vectors $\zeta_i$.
%What is missing to make it a zippered rectangle is a condition on imaginary parts
%that correspond the equation $(S_\pi)$ in~\cite{Yoccoz} on the suspension vector. For
%our purpose, this condition is useless and let us ignore it for the moment.
%\begin{figure}[!ht]
%\begin{center}\includegraphics{pictures/MasurPolygon.pdf}\end{center}
%\caption{A zippered rectangle translation structure with combinatorial data
%$\sigma_v = (0,7,14,1,16,11,4,17)$ or
%$\sigma_e = (0,16)(1,17)(7,11)(4,14)$ and
%$\sigma_f = (0,1,4,7,16,17,14,11)$. Here $\cE^{top} = \{0,11,14,17\}$ and $\cE^{bot} = \{4,7,16,17\}$. It is depicted as a sub-translation structure of an embedded triangulation.}
%\label{fig:TriangulationYoccoz}
%\end{figure}
%
%The triangulations in Figure~\ref{fig:TriangulationYoccoz} is
%\begin{align*}
%\sigma_v &= (0,3,6,9,7,5,14,15,1,16,13,10,8,11,12,4,2,17) \\
%\sigma_e &= (0,16)(1,17)(2,3)(4,14)(5,6)(7,11)(8,9)(10,12)(13,15) \\
%\sigma_f &= (0,1,2)(3,4,5)(6,7,8)(9,10,11)(12,13,14)(15,16,17)
%\end{align*}
%Here is a code sample to it with \cite{surface-dynamics}
%\begin{verbatim}
%sage: from surface_dynamics import FatGraph
%sage: edges = '(0,16)(1,17)(2,3)(4,14)(5,6)(7,11)(8,9)(10,12)(13,15)'
%sage: faces = '(0,1,2)(3,4,5)(6,7,8)(9,10,11)(12,13,14)(15,16,17)'
%sage: FatGraph(ep=edges, fp=faces)
%FatGraph('(0,3,6,9,7,5,14,15,1,16,13,10,8,11,12,4,2,17)',
%'(0,16)(1,17)(2,3)(4,14)(5,6)(7,11)(8,9)(10,12)(13,15)', 
%'(0,1,2)(3,4,5)(6,7,8)(9,10,11)(12,13,14)(15,16,17)')
%\end{verbatim}
%
%Now, given a topological zippered rectangle $(\sigma_v, \sigma_e, \sigma_f)$ we describe
%how to construct the bijections $\pi_t:\cA \to \{1,\ldots,d\}$ and $\pi_b: \cA \to \{1,\ldots,d\}$ of~\cite{Yoccoz}.
%Let $\cA$ be the set of the $d$ transpositions that appear in the cycle decomposition
%of $\sigma_e$ (these are the edges of the combinatorial map). Then by the second
%condition, $\cE^{top}$ and $\cE^{bot}$ are in canonical bijections with $\cA$: to an
%half-edge $i$ associates the unique transposition in $\cA$ that contains $i$.
%Next, there is a unique way to write $\sigma_f$ as
%\[
%\sigma_f = (b_1, b_2, \ldots, b_d, t_d, t_{d-1}, \ldots, t_1)
%\]
%where $\cE^{bot} = \{b_1, \ldots, b_d\}$ and $\cE^{top} = \{t_1, \ldots, t_d\}$.
%Then, for $\alpha = (t_i, b_j) \in \cA$ we set $\pi_t(\alpha) = i$ and $\pi_b(\alpha) = j$.
%
%In the example of Figure~\ref{fig:TriangulationYoccoz}, we have
%\[
%t_1 = 0,\ t_2 = 11,\ t_3 = 14,\ t_4 = 17
%\qquad
%b_1 = 1,\ b_2 = 4,\ b_3 = 7,\ b_4 = 16
%\]
%
%\subsection{Naive version}
%\textit{(discussed on April 22)}
%
%Pick an edge and build the first return map to this edge.
%For that purpose, follow the (vertical) translation flow until it comes back to this edge.
%


%The first step is to build a multi-interval exchange
%transformation out of a triangulation. Each triangle
%will correspond to an interval of the transformation
%and each edge will correspond to a subinterval of the
%transformation.
%
%Each triangle $(a,b,c)$ gives rise of two possible type
%of interval. Either
%\[
%\begin{array}{l}
%b\ a \\
%c
%\end{array}
%\qquad \text{or} \qquad
%\begin{array}{l}
%a \\
%b\ c
%\end{array}
%\]
%We call the first kind bottom-dominant and the second top-dominant.
%
%For each of these intervals, we can forget about the big edge and glue it
%to where it is glued. That diminish the number of intervals under consideration. In
%geometric terms, it corresponds to remove some of the edges of the triangulations.
%
%Another possible bootstrap option: start at a given vertex and look at the horizontal in the right
%direction. We start crossing triangles and we stop at the time we meet a triangle that we already
%encountered. Then remove all edges that are crossed by this horizontal piece. This makes a nice "Masur polygon".
%And we should start again from another vertex until we get rid of all the triangles.
%

\begin{thebibliography}{sage-flatsurf}

\bibitem[sage-flatsurf]{sage-flatsurf}
V. Delecroix, P. Hooper
\url{https://github.com/videlec/sage-flatsurf}

\bibitem[surface\_dynamics]{surface-dynamics}
V. Delecroix
\texttt{surface\_dynamics}

\bibitem[veerer]{veerer}
M. Bell, V. Delecroix, S. Schleimer
\texttt{veerer}

\bibitem[Yoccoz]{Yoccoz}
Jean-Christophe Yoccoz
\textit{Interval exchange maps and translation surfaces}
in Homogeneous flows, moduli spaces and arithmetic.
Proceedings of the Clay Mathematics Institute summer school,
(2010).
\end{thebibliography}

\end{document}
