\documentclass{article}

\usepackage{amsmath,amssymb}
\usepackage{graphicx}
\usepackage{url}
\usepackage[margin=2cm]{geometry}
\usepackage[abbreviations]{foreign}
\usepackage{xcolor}

\def\cA{\mathcal{A}}
\def\cE{\mathcal{E}}
\def\CC{\mathbb{C}}

\def\Re{\operatorname{Re}}
\def\Im{\operatorname{Im}}

\newtheorem{definition}{Definition}
\newtheorem{remark}[definition]{Remark}
\newtheorem{lemma}[definition]{Lemma}

\title{Algorithms for translation surfaces}
\author{Vincent Delecroix for Julian R\"uth}

\newcommand{\commv}[1]{{\color{red!50!gray}[#1]}}	

\begin{document}
\maketitle

\section{Combinatorial maps and translation structures}
\label{sec:TriangulationsAndTranslationStructures}

A cell decomposition of a surface can be encoded by a triple of
permutations.
\begin{definition}
Let $\cE$ be a finite set.
A \emph{combinatorial map} or a \emph{fat graph} on $\cE$ is a triple
of permutations $(\sigma_v, \sigma_e, \sigma_f)$ on
the set $\cE$ such that
\begin{enumerate}
\item $\sigma_f \circ \sigma_e \circ \sigma_v = id_\cE$,
\item $\sigma_e$ is an involution without fixed points,
\item the group generated by $\sigma_v, \sigma_e, \sigma_f$ acts transitively on $\cE$.
\end{enumerate}
\end{definition}
In the definition the letters  $v$, $e$ and $f$ respectively stand for vertices, edges
and faces.  Each cycle of the cycle decomposition of $\sigma_v$, $\sigma_e$
and $\sigma_e$ represents respectively a vertex, an edge
and a face of the cell decomposition. In particular, the cardinality of $\cA$
is even. The transitivity of the group ensures the connectivity of the surface.

\commv{tell me if you need more precision here about the topological viewpoint}

A particular case of combinatorial maps are \emph{triangulations}
when the permutation $\sigma_f$ is a product of respectively disjoint 3-cycles.
In these situation $n$ is a multiple of $6$. Translation structure are
easier to defined on triangulations.
\begin{definition}
\label{def:FlatTriangulation}
Let $T = (\sigma_v, \sigma_e, \sigma_f) \in S_\cE$ be a 
triangulation, a \emph{translation structure} on $T$
is a family of non-zero complex numbers $\{\zeta_i\}_{i\in\cE}$
such that
\begin{enumerate}
\item for each cycle $(i,j)$ of $\sigma_e$ we have $\zeta_i + \zeta_j = 0$
\item for each cycle $(i_1,i_2,i_3)$ of $\sigma_f$ 
the vectors $\zeta_{i_1}$, $\zeta_{i_2}$, $\zeta_{i_3}$ form the
sides of a triangle in $\CC$ where the sides appear in counter-clockwise order
(in particular $\zeta_{i_1} + \zeta_{i_2} + \zeta_{i_3} = 0$).
\end{enumerate}
\end{definition}
A translation surface is a geometrical notion and using triangulations
and the above definition is very often not enough.
For example in~\cite{veerer}, we need a more restrictive version
(related to the slopes of the vectors $\zeta_i$).
On the other hand, one might want to generalize translation structures
to cell decompositions that are not triangulations. An example of
this is considered in the next section with the zippered rectangle
construction.

For now, let us consider a simple definition that is not the most
general possible but which is simple to state.
\begin{definition}
\label{def:EmbeddedTranslationStructure}
Let $M = (\sigma_v, \sigma_e, \sigma_f) \in S_\cE$ be a combinatorial
map. An \emph{embedded translation structure} for $M$ is a zet of non-zero
complex numbers $\{\zeta_i\}_{i=1,\ldots,2n}$ such that
\begin{enumerate}
\item for each cycle $(i,j)$ of $\sigma_e$ we have
$\zeta_i + \zeta_j = 0$
\item for each cycle $(i_1,i_2,\ldots,i_k)$ of $\sigma_f$ the vectors
$\zeta_{i_1}$, \ldots, $\zeta_{i_k}$ form the sides of a $k$-gon in
$\CC$ where the vectors appear in counter-clockwise order.
\end{enumerate}
\end{definition}

\begin{remark}
When the $\zeta_i$ do not form a polygon it is still
possible to define a unique translation structure if and
only if the index around each face is $+1$ (where this index is measured in terms
of the arguments of the vectors). For now, let me avoid making
a formal definition but just say that for triangulations
this is equivalent to Definition~\ref{def:FlatTriangulation}.
\end{remark}

A \emph{sub-translation structure} is one that can be obtained by forgetting
some edges. Because a polygon can always be triangulated we have the following.
\begin{lemma}
Given a combinatorial map and a translation structure on it $(M,\zeta)$, there
exists a triangulation with a translation structure $(T,\zeta')$ so that $M$ is
a sub-translation structure of $T$.

Conversely, given a triangulation with a translation structure, this translation
structure restricted to any sub cell decomposition is a (not necessarily embedded)
translation structure.
\end{lemma}

\subsection{From translation structures to interval exchange transformation}
A zippered rectangle is a form of cell decomposition of a translation
surface into rectangles. It can be viewed as a particular kind of translation
structure as discussed in the previous Section. Though, as we will see, this
is not necessarily embedded.

\subsection{What is a zippered rectangle}
\begin{definition}
Let $\cE$ be a finite set of cardinality $2d$.
A \emph{topological zippered rectangle} is a combinatorial map $(\sigma_v, \sigma_e, \sigma_f) \in S_\cE$ such that
\begin{enumerate}
\item $\sigma_f$ is a $2d$-cycle (that is the cell decomposition consists of a single cell) and there is a partition of $\cE^{top} \cup \cE^{bot} = \cE$ such
that the involution $(\sigma_f)^d$ exchanges $\cE^{top}$ with $\cE^{bot}$.
\item $\sigma_e$ is such that each transposition $(i,j)$ that appears in its cycle
decomposition is such that one element is in $\cE^{top}$ and one element is in $\cE^{bot}$.
\end{enumerate}
A \emph{weak zippered rectangle translation structure} for $(\sigma_v, \sigma_e, \sigma_f)$ is
a set of complex non-zero vectors $\zeta_i$ such that
\[
\forall i \in \cE^{bot}, \Re(\zeta_i) \geq 0
\qquad
\forall i \in \cE^{top}, \Re(\zeta_i) \leq 0.
\]
\end{definition}
\begin{figure}[!ht]
\begin{center}\includegraphics{pictures/MasurPolygon.pdf}\end{center}
\caption{A zippered rectangle translation structure with combinatorial data
$\sigma_v = (0,7,14,1,16,11,4,17)$ or
$\sigma_e = (0,16)(1,17)(7,11)(4,14)$ and
$\sigma_f = (0,1,4,7,16,17,14,11)$. Here $\cE^{top} = \{0,11,14,17\}$ and $\cE^{bot} = \{4,7,16,17\}$. It is depicted as a sub-translation structure of an embedded triangulation.}
\label{fig:TriangulationYoccoz}
\end{figure}
The triangulations in Figure~\ref{fig:TriangulationYoccoz} is
\begin{align*}
\sigma_v &= (0,3,6,9,7,5,14,15,1,16,13,10,8,11,12,4,2,17) \\
\sigma_e &= (0,16)(1,17)(2,3)(4,14)(5,6)(7,11)(8,9)(10,12)(13,15) \\
\sigma_f &= (0,1,2)(3,4,5)(6,7,8)(9,10,11)(12,13,14)(15,16,17)
\end{align*}
Here is a code sample to generate the combinatorial triangulation from~\ref{fig:TriangulationYoccoz} with \cite{surface-dynamics}
\begin{verbatim}
sage: from surface_dynamics import FatGraph
sage: edges = '(0,16)(1,17)(2,3)(4,14)(5,6)(7,11)(8,9)(10,12)(13,15)'
sage: faces = '(0,1,2)(3,4,5)(6,7,8)(9,10,11)(12,13,14)(15,16,17)'
sage: FatGraph(ep=edges, fp=faces)
FatGraph('(0,3,6,9,7,5,14,15,1,16,13,10,8,11,12,4,2,17)',
'(0,16)(1,17)(2,3)(4,14)(5,6)(7,11)(8,9)(10,12)(13,15)', 
'(0,1,2)(3,4,5)(6,7,8)(9,10,11)(12,13,14)(15,16,17)')
\end{verbatim}

We now describe how to use the permutation $\sigma_f$ and the sets $\cA^{top}$ and $\cA^{bot}$
to the bijections $\pi_t:\cA \to \{1,\ldots,d\}$ and $\pi_b: \cA \to \{1,\ldots,d\}$ of~\cite{Yoccoz}.
Let $\cA$ to be the set of the $d$ transpositions that apper in the cycle decomposition
of $\sigma_e$. Then $\cA^{top}$ and $\cA^{bot}$ are in canonical bijections with $\cA$.
Next, there is a unique way of writing $\sigma_f$ as
\[
\sigma_f = (b_1, b_2, \ldots, b_d, t_d, t_{d-1}, \ldots, t_1)
\]
where $\cA^{bot} = \{b_1, \ldots, b_d\}$ and $\cA^{top} = \{t_1, \ldots, t_d\}$.
Then $\pi_t(\alpha)$ is simply the index $i$ such that $t_i$ appears in the cycle $\alpha$.

In the example of Figure~\ref{fig:TriangulationYoccoz}, we have
\[
t_1 = 0, t_2 = 11, t_3 = 14, t_4 = 17
\quad
b_1 = 1 b_2 = 4, b_3 = 7, b_4 = 16
\]

\subsection{The algorithm}
Given a translation structure on a triangulations as in
Section~\ref{sec:TriangulationsAndTranslationStructures} we
want to produce a finite union of interval exchange transformations.
The algorithm consists of three steps
\begin{enumerate}
\item horizontal exploration,
\item eliminating mixed and large edges,
\item reduction (or generalized Rauzy induction).
\end{enumerate}

%The first step is to build a multi-interval exchange
%transformation out of a triangulation. Each triangle
%will correspond to an interval of the transformation
%and each edge will correspond to a subinterval of the
%transformation.
%
%Each triangle $(a,b,c)$ gives rise of two possible type
%of interval. Either
%\[
%\begin{array}{l}
%b\ a \\
%c
%\end{array}
%\qquad \text{or} \qquad
%\begin{array}{l}
%a \\
%b\ c
%\end{array}
%\]
%We call the first kind bottom-dominant and the second top-dominant.
%
%For each of these intervals, we can forget about the big edge and glue it
%to where it is glued. That diminish the number of intervals under consideration. In
%geometric terms, it corresponds to remove some of the edges of the triangulations.
%
%Another possible bootstrap option: start at a given vertex and look at the horizontal in the right
%direction. We start crossing triangles and we stop at the time we meet a triangle that we already
%encountered. Then remove all edges that are crossed by this horizontal piece. This makes a nice "Masur polygon".
%And we should start again from another vertex until we get rid of all the triangles.
%

\begin{thebibliography}{sage-flatsurf}

\bibitem[sage-flatsurf]{sage-flatsurf}
V. Delecroix, P. Hooper
\url{https://github.com/videlec/sage-flatsurf}

\bibitem[surface\_dynamics]{surface-dynamics}
V. Delecroix
\texttt{surface\_dynamics}

\bibitem[veerer]{veerer}
M. Bell, V. Delecroix, S. Schleimer
\texttt{veerer}

\bibitem[Yoccoz]{Yoccoz}
Jean-Christophe Yoccoz
\textit{Interval exchange maps and translation surfaces}
in Homogeneous flows, moduli spaces and arithmetic.
Proceedings of the Clay Mathematics Institute summer school,
(2010).
\end{thebibliography}

\end{document}
